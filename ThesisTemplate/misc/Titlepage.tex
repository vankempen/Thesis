\begin{titlingpage}
\linespread{1}\selectfont

\BgThispage
\newgeometry{left=1cm,right=4cm}
\vspace*{2cm}
\noindent
\textcolor{white}{\bigsf Airfare lock-in products}
\vspace*{1.7cm}\par
\noindent
\begin{minipage}{0.35\linewidth}
    \begin{flushright}
    \underline {\small \textit student}\vspace{0.3ex}\\
    P.N. van Kempen \\
    322165 \\
    Supply Chain Management\vspace{3ex} \\
    \underline {\small \textit coach}\vspace{0.3ex}\\
    N.A.H. Agatz\vspace{1.5ex} \\
    \underline {\small \textit co-reader}\vspace{0.3ex}\\
    \ldots\vspace{3ex}\\
    \underline {\small \textit date}\vspace{0.3ex}\\
    \today\\
    \end{flushright}
\end{minipage} \hspace{15pt}
%
\begin{minipage}{0.02\linewidth}
    \rule{1pt}{175pt}
\end{minipage} \hspace{-10pt}
%
\begin{minipage}{0.6\linewidth}
\vspace{5pt}
    \begin{abstract}
In this master's thesis I want to research whether it is sustainable for an external company without capacity information or seat reservation capabilities to set the price of airfare~lock-in~products at a level that the customer accepts. An airfare~lock-in product gives its holder the right --- but not the obligation --- to buy the underlying flight ticket for a predetermined price on or before the maturity date. The research questions will be tested using a theoretical framework and empirical data gathered on 22~routes for a period of 12~weeks. The acquired data will also function as the census for the two option valuation methods (i.e. \emph{Black--Scholes}, and \emph{Monte Carlo}) and the simulation model to assess the external company's performance. Lastly, sensitivity analysis on the properties associated with the option buyer -- \emph{risk utility}, \emph{forecast accuracy}, and \emph{likelihood of traveling} --- is carried out to see the impact of these aspects on the returns of the option seller.
    \end{abstract}
\end{minipage}
\end{titlingpage}
\restoregeometry