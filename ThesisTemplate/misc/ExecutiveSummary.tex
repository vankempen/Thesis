\renewcommand*\abstractname{Executive summary}
\begin{abstract}
Airlines have to operate on low profit margins while dealing with extreme risks and uncertain demand. To cope with these challenges, the aviation industry has implemented revenue management models. The models allowed companies to better segment their customers and make more accurate estimates of a passenger's maximum willingness to pay. While this has lead to large increases in revenues for the airliners, customers have to deal with much higher levels of uncertainty. Airfares might jump hundreds of dollars overnight, and insuring yourself for possible cancellation is many times more expensive than the cheapest airfare.

In this master's thesis, I want to study the concept of \emph{airfare~lock-in products} and test whether it is viable for an external company without capacity information to sell this new type of goods. Airfare~lock-in products allow the customer to fixate the price of a seat for a predetermined period at a certain premium. The customer can thus decide at a later moment in time whether to actually buy the ticket or not, while covering himself from price increases. Furthermore, when an airfare drops a large amount, the customer can still choose to not execute the product and buy the ticket at the lower rate. An airfare~lock-in product therefore also allows the customer to gain from price decreases. This concept is thus comparable to call options in financial markets.

I will research the viability of offering such options by training and testing historical data. Data on 22~routes is collected for flights departing in a period from August $19^{th}$ till September $30^{th}$. Each flight's fare is checked 4~times per day for a period of 6~weeks before departure. The first 4~weeks of the census (i.e., August $19^{th}$ -- September $16^{th}$) will be used to train two option valuation models. The first method will base its predictions using the Black--Scholes method, while the other approximates the best option price using Monte Carlo. After the creation of the models a simulation will be run based upon the last two weeks of the historical data. This simulation tests the profitability of the two models and compares them with the most optimal situation. It will analyse whether it is theoretically and practically possible for an external company that sells the options to sustain. Furthermore, the simulation will also conclude which of the option valuation models approximates the optimal option price best.

After the simulation a sensitivity analysis will be performed as well. In this analysis the three~parameters of a passenger --- \begin{inparaenum}[\itshape (i)\upshape]
\item risk-utility,
\item forecasting technique, and
\item likelihood of travelling
\end{inparaenum} --- will be altered to see what the effect of these properties is on the profitability of the company that offers airfare~lock-in products.
\end{abstract}