\chapter{Executive summary}
Airlines have to operate on low profit margins while dealing with extreme risks and uncertain demand. To cope with these challenges, the aviation industry has implemented revenue management models. These models allowed companies to better segment their customers and make more accurate estimates of a passenger's maximum willingness to pay. While this has led to large increases in revenues for the airliners, customers have to deal with much higher levels of uncertainty. Airfares might jump hundreds of dollars overnight, and insuring yourself for possible cancellation is many times more expensive than the cheapest airfare.

In this master's thesis, I want to study the concept of \emph{airfare~lock-in products} and test whether it is feasible for an external company to sell this new type of goods in a sustainable manner. Airfare~lock-in products allow the customer to fixate the price of a seat for a predetermined period of time at a certain premium. The customer can thus decide at a later moment in time whether to actually buy the ticket or not, while covering himself from price increases. Furthermore, when an airfare drops a large amount, the customer can still choose to not execute the product and buy the ticket at the lower rate. An airfare~lock-in product therefore also allows the customer to gain from price decreases. This concept is thus comparable to call options in financial markets.

This study will consist of two consecutive parts: data collection, and simulation with empirical analysis.

The first part of this research will concern itself with the collection of historical data. These data on 22~routes are collected for flights departing in a period from August $19^{th}$ till September $30^{th}$. Each flight's fare is checked four~times per day for a period of six~weeks before departure. The first four~weeks of the census (i.e. August $19^{th}$ -- September $16^{th}$) will serve as the training set. This set is being used by the customer as the base for her weak-form ration expectations. Furthermore, the option company will use the data to make predictions of future price movements using Monte~Carlo simulation. The latter two~weeks of the acquired data will serve as the test set to empirically test the simulation models.

The second part of this thesis will engage in the empirical analysis of the collected data. It will furthermore run several simulations to test the performance of different configurations and provide answers to the research questions.

The main research objective of this thesis is to determine whether it is sustainable for an external company without capacity information or seat reservation capabilities to set the price of airfare~lock-in~products at a level that the customer accepts. To provide the answer to this question, I first construct a theoretical model. The analysis shows that it is indeed possible to offer options at a fee that both the seller and customer would accept. However, in the hypothetical model the company would never be able to make any profits on selling lock-in products due to the configuration of the all knowing passenger. To make this model practically implementable, some of its assumptions were dropped, and another simulation was performed. The new model, which assumes the customer had less accurate forecasting information than the option company, actually shows that it is possible to realize profits by offering options. Another simulation that implements a practical forecasting method (i.e. Monte~Carlo simulation) reveals that obtaining positive returns on selling lock-in products is even possible with the help of real world applications.

The next finding from the analysis of the simulation models in this research is that passengers will in some situations also benefit from purchasing an option. Without the availability of an option, the customer has to decide whether she will buy the ticket immediately or wait a few extra days to gain some extra decision time. Due to uncertainty, however, the customer might take the wrong decision, which results in suboptimal results.

This thesis presents two~cases in which the decision to buy an airfare lock-in product will be different from the outcomes of the actual decision if no options would be available. The analysis of the first case, in which the customer would buy an option when she instead should have bought the flight immediately, shows that the customer will be worse off by purchasing the lock-in product. In the other case, in which the overestimating passenger would make the wrong decision to buy the flight instead of waiting, the purchase of an option actually results in positive outcomes for the customer.

In this study, I defined three~strategies of option purchasing: the first assumes the customer will never buy an option, the second strategy assumes the passenger will always purchase the lock-in product, and the third methodology assumes the customer will either decide to buy an option or not. The third strategy turned out to be almost always preferable relative to the first strategy. Furthermore, the cost savings will become higher as the inaccuracy of the customer's forecasts progresses. Lastly, the analysis shows that after a certain level of the passenger's error term, the second strategy will even outperform the methodology of never buying an option.

The final research question in this thesis concerns itself with the influence of the factors \emph{seller's forecasting accuracy}, his \emph{applied margin}, and the \emph{customer's forecasting accuracy} on the performance of the model. To determine the effects of the parameters, I will run several more simulations. The analysis on the influence of the seller's accuracy will shows that not the absolute forecasting error is important for company's performance, but rather the level of underestimation of its predictions. The simulation on the impact of different levels of the applied margin will reveal that this parameter is a significant tool for performance when the seller has lower forecasting accuracy, as it can compensate for these errors. Lastly, the customer's forecasting accuracy turns out to be of importance, as erroneous predictions will result in the passenger making the wrong buy/wait decision more often. The cost savings when purchasing an option would lower the damages done by this wrong decision.