\chapter{Executive summary}
Airlines have to operate on low profit margins while dealing with extreme risks and uncertain demand. To cope with these challenges, the aviation industry has implemented revenue management models. These models allowed companies to better segment their customers and make more accurate estimates of a passenger's maximum willingness to pay. While this has led to large increases in revenues for the airliners, customers have to deal with much higher levels of uncertainty. Airfares might jump hundreds of dollars overnight, and insuring yourself for possible cancellation is many times more expensive than the cheapest airfare.

In this master's thesis, I want to study the concept of \emph{airfare~lock-in products} and test whether it is viable for an external company without capacity information or reservation capabilities to sell this new type of goods in a sustainable manner. Airfare~lock-in products allow the customer to fixate the price of a seat for a predetermined period of time at a certain premium. The customer can thus decide at a later moment in time whether to actually buy the ticket or not, while covering himself from price increases. Furthermore, when an airfare drops a large amount, the customer can still choose to not execute the product and buy the ticket at the lower rate. An airfare~lock-in product therefore also allows the customer to gain from price decreases. This concept is thus comparable to call options in financial markets.

I will research the viability of offering such options by running several simulations and performing empirical analysis with the help of historical data. These data on 22~routes are collected for flights departing in a period from August $19^{th}$ till September $30^{th}$. Each flight's fare is checked 4~times per day for a period of 6~weeks before departure. The first 4~weeks of the census (i.e. August $19^{th}$ -- September $16^{th}$) will serve as the training set. This set is being used by the customer and the option seller as the base for the weak-form ration expectations. Furthermore,

