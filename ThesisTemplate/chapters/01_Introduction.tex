\chapter{Introduction}
\label{chap:Introduction}
\chapterprecishere{The fastest way to become a millionaire is to start off with a billion dollars and invest in an airline}

It might not come as a surprise that companies in the aviation sector have to operate on profit margins close to none\footnote{\url{http://on.wsj.com/MftvqH}}, and that they were struck hard by the financial crisis. To cope with these challenges, airliners are known to implement many revue management concepts (e.g. market segmentation, protection levels). Robert Crandall, former CEO of American Airlines, even said stated that yield management is ``\emph{the single most important technical development in transportation management since we entered deregulation}'' \cite[p.~30]{cross97}.

The implementation of such revenue management models has caused major changes in the aviation sector. For instance, the application of booking protection levels has created many fluctuations in airfares within a single flight. Early bookers usually can purchase a ticket at a low rate while passengers later in the booking process can only acquire the same seat at a much higher fare. Early booking of a flight, however, brings along much more uncertainty, as there might occur many situations in which the buyer is unable to utilize the ticket. For example, when a passenger wants to go skiing, but breaks his leg prior to departure. A person can cover himself from any losses caused by these kind of situations by buying insurance. However, insurance does not always provide a solution for no-shows. By way of illustration, consider you would like to go on holiday with a group of your friends, but your friends are undecided over whether they can go on the trip or not. Now you have two options to choose from:
\begin{inparaenum}[\itshape (i)\upshape]
    \item buy the flight now and risk not using this ticket because your friends can't come along, or
    \item wait to buy the flight until you know for sure your friends will go and risk sudden price increases in airfare or even sell-outs.
\end{inparaenum}
This extra level of uncertainty is undesired by customers, and makes it more difficult to make rational decisions.

A recent trend in the aviation industry --- the availability of \emph{airfare lock-in products} --- might provide a solution to this problem of uncertainty. An airfare lock-in product allows a customer who is unsure whether he wants to directly buy a ticket to buy an \emph{option} on that flight instead. The option gives him the opportunity to buy the actual ticket at a later time at a predetermined price. Some airliners (e.g., United Airlines, Air France-KLM, Estonian Air) already offer this new type of service and allow the customer to reserve a seat for 2 to 14~days. New external companies like OptionIt, BitBend, SteadyFare, also arise, though all still in their initial stage\footnote{OptionIt currently offers only options for sporting events, but plans to offer options for tickets of major airlines. BitBend and SteadyFare are both not available for public use}.

The offering of lock-in products might be viable to passengers and external companies that write these options, as well as the actual airlines. However, research on this topic is scarce and results of successful applications of the strategy are not yet public. Further theoretical and empirical research might provide answers to the actual viability, and could offer support for companies that would like to implement this type of product. Hence, this is the topic my thesis is about.
