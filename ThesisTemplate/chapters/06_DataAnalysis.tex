\chapter{Data analysis}
\label{chap:DataAnalysis}
\section{Data description}
Flight ticket prices were collected 4~times per day for a period ranging from the $9^{th}$ of July till the $29^{th}$ of September. During this data collection phase, airfares on the 22~routes described in \autoref{app:SelectedRoutes} were gathered on flights departing from the $20^{th}$ of August till the $30^{th}$ of September. In these 12~weeks, 158,553~files were requested from Google~Flights' RPC-servers, which resulted in a set of almost 18~gigabites of JSON~data. 15~of these requests failed at the first try, and had to be downloaded again one quarter of an hour later. Of these retries, the automated script also failed to download 2~files at the second request. These failures were the request made on the $20^{th}$ of September at 00:00 GMT+2 for flights departing from JFK to LHR on the $25^{th}$ of September and the request made on the $26^{th}$ of July at 00:00 GMT+2 for flights departing from LHR to LAX on the $1^{st}$ of July. As the next paragraphs will explain, these two~failures did not create any gaps of missing data in the final data structure.

After retrieval of the data all the JSON-files were parsed for single-legged return flights and categorized by outbound/inbound airport, and operating carrier. Each flight was distinguished using the characteristics described in \autoref{subsec:FlightTicket}, and departure times were converted to UTC\@. Next, the timeslot $p_{t_n}$ of the lookup time relative to the actual departure time was determined for each flight, where $n = \lfloor \frac{\scriptstyle{hours\ before\ departure}}{6} \rfloor + 1$. Because files were downloaded for 4~times per day, they can be assigned to timeslots consisting of 6~hours. For example, an airfare retrieved 33 hours before departure of the flight will be assigned to timeslot $p_{t_6}$, where $6 = \lfloor \frac{33}{6} \rfloor + 1$. Because ticket prices were acquired from 1~day up until 42~days before departure, a total of 168~timeslots are available ranging from $p_{t_5}$ till $p_{t_{172}}$.

After parsing of the files, the prices of the 6~hour timeslots were converted to represent daily values. To do so, every 4~timeslots where thus aggregated into a single airfare by averaging the values. For instance, the timeslots $p_{t_{17}}$, $p_{t_{18}}$, $p_{t_{19}}$ and $p_{t_{20}}$ were averaged into a single value $p_{d_4}$ representing the ticket price for that specific flight as available 4~days before departure.

\noindent
An illustration of this process is given in \autoref{fig:conversionOfTimeslotsIntoDailyAirfares}.

\begin{figure}
$$
\kbordermatrix{
           &\multispan4\hfill$\scriptstyle 1\,dbd$\hfill    &        \\
    \omit  &\multispan4\quad\downbracefill\quad             &        \\
    \#1    & p_{t_5} & p_{t_6} & p_{t_7} & p_{t_8} & \vrule & \cdots \\
    \#2    & p_{t_5} & p_{t_6} & p_{t_7} & p_{t_8} & \vrule & \cdots \\
    \vdots & \vdots  & \vdots  & \vdots  & \vdots  & \vrule & \ddots \\
    \#n    & p_{t_5} & p_{t_6} & p_{t_7} & p_{t_8} & \vrule & \cdots 
}
\Rightarrow
\kbordermatrix{
           & 1\,\scriptstyle{dbd}       & \cdots & 42\,\scriptstyle{dbd}              \\
    \#1    & \overline{p_{t_{5,6,7,8}}} & \cdots & \overline{p_{t_{169,170,171,172}}} \\
    \#2    & \overline{p_{t_{5,6,7,8}}} & \cdots & \overline{p_{t_{169,170,171,172}}} \\
    \vdots & \vdots                     & \ddots & \vdots                             \\
    \#n    & \overline{p_{t_{5,6,7,8}}} & \cdots & \overline{p_{t_{169,170,171,172}}}
}
$$
\caption{Conversion of timeslots into daily airfares}
\label{fig:conversionOfTimeslotsIntoDailyAirfares}
\end{figure}

When certain timeslots contain no price observation, this single timeslot is left out of the calculation of the average airfare for that specific number of days before departure. Only when all 4~of the timeslots for a specific day have no data available, the value in the new data matrix is assigned as \emph{missing value} (i.e., n/a). Due to this methodology in which the daily 4~data~points are aggregated into a single value, the 2~failures described earlier do not create any gaps of missing values in the final data matrix. \todo{analysis}

The conversion into daily values resulted in a database containing 462,662 unique flights which included 12,865,953 airfare observations. Due to Google Flights' nature of returning a limited set of airfares for each request, some flights did contain gaps of missing values within their data. In alignment with the study by \citeA{groves2013agent}, this paper therefore excludes flights for which less than $\frac{2}{3}$ of the possible airfare observations are available. So, because there are a total of 42~observations days prior to each flight, tickets that do have fewer points than $42 \times \frac{2}{3} = 28$ available will be excluded from the analysis. Next, because airlines are known to open and close specific price-buckets \cite{mcgill1999revenue}, some gaps were still present in the dataset. These gaps were filled with the data available from observations made in the future. For example, when a particular flight misses a price observation at 35~days before departure, but it does contain the airfare for 34 and 36~days prior to departure, the 35~days' slot would be filled with the same value as the 34~days before departure. In reality this would mean that when a customer's option expires, but the ticket is not available on the date of maturity (i.e., no price observation). The customer thus would have to wait till the airline resumes the sale of that flight. He would therefore get the ticket at the price in the future.

Missing data at for the last (few) days prior to departure are considered to be the sell-out of a flight. Because Google Flight's also offers Business and First class tickets, this implies that all the seats are completely sold out, and there is no possibility of still buying the ticket. This means that when a customer's option expires on such a ticket and he wishes to exercise it, an alternative flight has to be offered. To prevent such situations from happening, these missing data on sold-out flights is filled with a penalty similar to the technique used by \citeA{etzioni03}. By introducing such a penalty, the option valuation model is triggered into trying not to offer options when the probability of sell-out is high. Instead of the \$300,000 proposed by the authors, I have chosen to set the penalty at 3 times the average airfare at 1~day before departure. This is more realistic in a sense that the customer could be offered an alternative flight that leaves on the same day plus some extra form of monetary compensation. Furthermore this lower fill-value prevents penalties from biasing the airfares closer to departure.

An illustration of this data filling process is given in \autoref{fig:dataFillingOfMissingValues}. Airfares are given as integers representing the price in cents. As can be seen in the figure, the n/a-gaps are filled with the nearest available future price closer to departure. Furthermore, the missing value for the flight with id~\#3 at 1~day before departure is seen as a sold-out ticket. This value is filled with $3 \times \bar{c_1}$, where $\bar{c_1}$ represents the average of the airfares observed at a single day before departure.

\begin{figure}
$$
\kbordermatrix{
           & 1\,\scriptstyle{dbd} & 2\,\scriptstyle{dbd} & 3\,\scriptstyle{dbd}  & \cdots \\
    \#1    & 10000                & 12000                & 12000                 & \cdots \\
    \#2    & 20000                & \mathrm{n/a}         & 22000                 & \cdots \\
    \#3    & \mathrm{n/a}         & 32000                & \mathrm{n/a}          & \cdots \\
    \vdots & \vdots               & \vdots               & \vdots                & \ddots
}
\Rightarrow
\kbordermatrix{
           & 1\,\scriptstyle{dbd} & 2\,\scriptstyle{dbd} & 3\,\scriptstyle{dbd}  & \cdots \\
    \#1    & 10000                & 12000                & 12000                 & \cdots \\
    \#2    & 20000                & 20000                & 22000                 & \cdots \\
    \#3    & 3 \cdot \bar{c_1}    & 32000                & 32000                 & \cdots \\
    \vdots & \vdots               & \vdots               & \vdots                & \ddots
}
$$
\caption{Data filling of missing values (i.e., n/a)}
\label{fig:dataFillingOfMissingValues}
\end{figure}

The empirical dataset shows that in total 159,600 flights are not offered (i.e., sold-out) at 1~day before departure. These tickets take into account almost 35 percent of the all the flights available. Real-world data on the number of sold-out flights is very hard to acquire, because this value could give sensitive information on the performance of an airline.  \todo{Kindervater?}


After cleaning the dataset, it was split into two separate parts. The database used in this research contains airfare observations of flights departing in the period $20^{th}$ of August till the $30^{th}$ of September. As explained in \autoref{chap:Methodology}, the first 4~weeks of these departing flights --- $20^{th}$ of August till the $16^{th}$ of September --- will be used as the training set for the research. The latter 2~weeks --- from the $17^{th}$ of September till the $30^{th}$ of September --- will be used for testing the proposed option valuation models. This split resulted in a training set containing 321,545 flights, and a test set which includes 141,117 flights. An example extract of the final test data matrix used for the flights from AMS to JFK can be found in \autoref{fig:extractFromAMStoJFK}. In this extract, flight with id~\#4 was sold-out 1~day before departure, and that field was thus filled with $3 \times$ the average ticket price at 1~day before departure.

\begin{figure}
$$
\kbordermatrix{
           & 1\,dbd & 2\,dbd & 3\,dbd & 4\,dbd & 5\,dbd & \ldots & 42\,dbd \\
    \#1    & 57469  & 68619  & 122469 & 222469 & 109644 & \ldots & 65853   \\
    \#2    & 87119  & 87369  & 93019  & 84469  & 79469  & \ldots & 65853   \\
    \#3    & 87494  & 86694  & 94744  & 74569  & 70244  & \ldots & 65853   \\
    \#4    & 410951 & 134430 & 242480 & 108934 & 126369 & \ldots & 123357  \\
    \vdots & \vdots & \vdots & \vdots & \vdots & \vdots & \ddots & \vdots  \\
    \#252  & 68984  & 69094  & 64669  & 73219  & 70244  & \ldots & 65857
}
$$
\caption{Sample extract from test data matrix AMS to JFK}
\label{fig:extractFromAMStoJFK}
\end{figure}

\noindent
A further analysis of these datasets, including more specific statistical numbers on the 22~routes, can be found in \todo{table ref}

\todo[Table]{table with analysis of sets}





%Future research
%passenger arrival rate to de Boer and Wetherford

%is it really bad? => surplus


%TODO
%- Table for total of flights per out/inbound dest, and operating carrier
%- Analysis of sold-out flights

