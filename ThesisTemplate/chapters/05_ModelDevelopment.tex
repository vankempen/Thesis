\chapter{Model development}
\label{chap:ModelDevelopment}
This chapter will discuss the simulation model used for the evaluation of the forecasting techniques. \todo{descr}

\section{Overview of simulation model}
\label{sec:SimulationModel}
In this thesis, I will run a series of simulations to determine the performance of different option valuation models. These pricing models range from theoretical optimal --- the seller with perfect foresight --- up till the practically implementable --- based upon historical prices and Monte~Carlo simulation.

The first simulation run will determine the maximum possible profits an option seller could realize when the customer uses strong-form rational expectations. The results will be acquired by assuming an option writer that has perfect information on price movements and customer characteristics, and thus acts accordingly. Though these assumptions are unrealistic in a real word setting, the observations of this run can be used as the benchmark to compare the performance of other option models to.

The runs after the optimal case will drop some of the assumptions made about the seller with perfect foreknowledge, and determine the influences of those assumptions on the results. The final case will be a practically implementable option valuation model based upon historical prices.

To determine the outcomes associated with each of the models, each simulation will run through a number of steps. This process is illustrated in \autoref{fig:simulationProcess}. The processes above the line are the ones associated with the buyer, the customer submodel. The ones below are connected to the airfare~lock-in product seller, the seller submodel.

\insertfigure{simulationProcess}{Illustration of simulation process}

The process shown in the figure can be described in more detail using the seven~consecutive steps:

\begin{description}
\item[Arrival of passenger] In the first stage of the simulation, the model will generate the arrival of passengers. A passenger is interested in buying an airfare~lock-in product on a specific flight. The simulation does not calculate the customers willingness to pay for the ticket price, but assumes that they all in principle want to buy the ticket at the current fare. Therefore, the model only determines whether the customer wants to buy the option on the flight.

In this research, I will generate passengers according to a homogeneous Poisson distribution with an average arrival rate of 1 customer per option per day before departure ($\sim \mbox{Pois}(\lambda=1)$). The use of a Poisson distribution is common in literature on simulating arrivals for flights (e.g., \citeA{gosavi2007simulation}, \citeA{bertsimas2005simulation}, \ldots).

Because data have been collected on flights up till 42~days before departure, each ticket will receive this same number of passengers with option requests on average. However, because when selling options with a specific number of days to maturity $m$ it is not possible to buy options fewer that $m$ days to departure, these customers get excluded from the model. As an example, the test dataset containing airfares of tickets from LHR to JFK includes 7,690 unique flights after cleansing. When simulating this model for options with a maturity of 3 days an expected total of $(42 - 3) \times 7,690 = 299,910$ passengers will be generated. Due to the probabilistic outcomes of the Poisson distribution, the exact number of arrivals number might vary.

\item[Calculate the passenger's WTP] The next step of the simulation will engage itself in the calculation of the customer's Willingness To Pay. This amount is determined by computing the minimum of the expected utility of buying the flight immediately, or postponing the decision to fly $m$~days later.

The implementation of the customer's WTP decision model is explained in detail in \autoref{sec:ModelPassengerBehaviour}, and the exact calculation of this WTP in \autoref{sec:CustomerSubmodel}.

Now that the customer knows his level of WTP, he requests the seller for the price of the option.

\item[Calculate the option seller's WTA] The third step is responsible for the computing the option seller's minimum \emph{Willingness To Accept}. A writer's WTA is the minimum level at which the seller wants to offer options that insure against price risks. When he would offer the airfare~lock-in products at this value, the seller would not expect any returns on the sale. The WTA of the external party is thus calculated using the writer's predictions of how the airfare is expected to change.

The simulation model in this thesis uses many different configurations for the option sellers. In the benchmark case, a lock-in product seller with perfect information is considered. His WTA for each option is the same as the observed level of increase of the underlying fares. The different methods for computing the WTA for every particular configuration of the writer will be described in detail in \autoref{sec:sellerSubmodel}.

\item[Calculate the price of the option] Step number four will calculate the price at which the option will be offered to the passenger. This simulation will calculate this value by applying a margin on top of the WTA calculated in previous step. To ensure that the seller does not offer options for a negative revenue --- which could be the case when he expects a decrease in airfare --- also a minimum price is set. All calculated option prices that fall below this set minimum price will be raised to this level. After computing the option price, the customer then gets offered the option at this fare.

\item[Acceptance of the offer] The next step will determine whether the customer will accept the airfare~lock-in product offer. This is done by comparing the customer's WTP and the offered option price. When the Willingness To Pay of the customer for a certain flight is higher than the option price calculated in prior process, a customer will accept the offer (i.e. $\text{WTP} \ge p_O$).

\item[Exercising the option] The second-last step in the simulation process will determine whether the customer will actually exercise his option at the date of maturity. To do so, the model will first `wait' $m$~days, and check the observed ticket price. A customer will only exercise her right when this observed airfare is higher than the strike price of the option, $S$. Furthermore, the passenger will only use the option when he has decided to fly. When both of these rules apply to a certain customer, he will thus make use of his lock-in product, and let the seller buy the flight ticket for him. The passenger's decision tree associated with this choice is given in \autoref{fig:exercise_option}.


\item[Calculate generated outcomes] The final step in the model will calculate the outcomes of each sold option. When the customer does not exercise his option on the date of maturity, the seller will gain the full price at which he has sold the option in profits. When the passenger actually does exercise her right, the profits or losses gained from selling the option can be defined as:
$$ \Pi = p_O - (S - T_m) $$
\end{description}

Using the method described above, the simulation will be used to compare the performance of the proposed option valuation models. The same procedure will also be used to test the influence of a particular configuration of parameters within a single option valuation model.


\section{Customer submodel}
\label{sec:CustomerSubmodel}
In this research, two different submodels are being considered; the \emph{customer submodel} and the \emph{seller submodel}. This section will describe the customer model. The next section will concern itself with the definition of the seller submodel.

The customer submodel concerns itself with the calculation of the willingness to pay on the passenger's side of the model. To compute his willingness to pay, the customer uses the following equation:

$$ \mbox{WTP} = V_c \times P^f $$

Where $P^f$ is the customer's likelihood of travelling and $V_c$ is the customer's expectation of the standard option price. In this thesis, \emph{standard option price} refers to the value of an option that a customer would be willing to pay when his $P^f$ is equal to 1.

The customer's expectation of the option price is calculated using a specific prediction method. In this research, the customer will base these forecasts on his \emph{rational expectations} (RE) of ticket price movements. This type of prediction method was first introduced by \citeA{muth1961rational}, and later generalized to also include more cases by \citeA{arrow1962economic}. The expectations generated from the RE-model assume that the entity makes rational predictions based upon all the information he has, plus a random \emph{error term}. In the strong case, the RE-model assumes the entity's predicted values are the actual observed prices plus a random fluctuation. The weak case, however, assumes the customer will base his forecasts only on available (historical) information, $I_0$. Because this information might differ from the actual observed values, this method is expected to be less accurate. A more detailed description of these two forms of expectations is given below.

\paragraph{1. Strong-form rational expectation (used in benchmark)}
In the strong-form case, the customer will base her expectations (i.e. $E[T_m]$) upon the actual price changes as seen in the future (i.e. $T_m$):

$$ E[T_m] = T_m \times \epsilon$$

The random error in the equation will be drawn from a normal distribution with mean 1 and a specified standard deviation:
$$ \epsilon \sim \mathcal{N} (1, \sigma^2) $$

A higher standard deviation will thus lower the accuracy of the passenger's expectations. For illustration, the connection can be observed in the relation between this error term and the $\mbox{MAPE}$. A detailed description of this relation is provided in \autoref{app:errorMAPE}.

The resulting prediction $E[T_m]$ will be a point estimate which represents the expected future airfare according to the customer. To compute his standard option price, the passenger will subtract the current ticket price from the acquired point estimate:

$$ E[V_c] = E[T_m] - T_0$$


\paragraph{2. Weak-form rational expectation}
When applying the weak-form case of rational expectation, the customer will base her expectations upon the historically observed changes:

$$ E[T_m] = (E[T_m] | I_0) \times \epsilon $$

Where $I_0$ represents the historical distribution of price ticket changes of the current route. In this thesis I thus assume the customer has on average the availability over this correct historical distribution.

Like in the strong-form RE, the random error term in the above equation will be drawn from a normal distribution with mean one and a standard deviation. A higher standard deviation also results in lower accuracy of the passenger's predictions.

From the historical distribution, this form of rational expectation will construct an expected distribution of price changes of the current airfare. Next, the expected distribution of fluctuations is used to calculate the anticipated ticket price at the option's date of maturity:

$$ E[T_m] | I_0 = \sum |T_m \times P[T_m]| $$

The customer's standard option price is then computed by subtracting the current airfare from the expected future fee:
$$ V_c = E[T_m] - T_0 $$

For the benchmark, the simulation assumes that the customer will base his expectations using the \emph{strong-form rational expectation} method. After the computation of the passenger's standard option price, his willingness to pay has to be determined using his probability of flying.

\paragraph{Passenger's likelihood of travelling}
The passenger's likelihood of travelling gives the probability that he will decide to fly when his option matures. Each passenger's $P^f$ is being simulated by drawing a random value from a uniform distribution between zero and one:

$$ P^f \sim \mbox{Unif}(0, 1)$$

Every customer thus has his own random value of $P^f$. The model assumes the customer knows the \emph{exact} level of this probability, and acts accordingly. As shown at the beginning of this section, this probability of flying influences the passenger's willingness to pay of an option linearly; a low $P^f$ results in a low likelihood of actually having to use the option, while a high probability leads to a higher likelihood of having to exercise the product at maturity. The customer's willingness to pay for an option thus is the standard option price times this probability.

Next, the customer will only accept the option if his willingness to pay is greater than the offer he receives from the seller:

$$ \mbox{WTP} \ge \mbox{WTA} $$

The methodology for how the seller sets the price of the option $\mbox{WTA}$ is described in detail in \autoref{sec:sellerSubmodel}.


\subsection{Exercise of the option}
On the date of maturity, a Bernoulli trial will be run to determine whether the customer will actually fly. When the outcome of the trial is $0$, the customer has decided to \emph{not} travel and thus make no use of the option. When the outcome of the trial is $1$, the customer has decided that he will make use of the flight.

\begin{equation}
\sim \mbox{Bernoulli}(P^f)\begin{cases}
     \mbox{fly}, & \mbox{if } 1 \\
    \mbox{don't fly}, & \mbox{if } 0 \end{cases}
\end{equation}

Due to this property, the passenger will thus always receive information on the date of maturity on whether she will fly.

After the Bernoulli trial, the customer will have to decide whether she will exercise his option. This is the case when the following applies:
\begin{compactitem}
    \item the ticket price on the option's date of maturity is greater than the strike price, ($T_m > S$) and
    \item the customer \emph{does} want to make use of the flight.
\end{compactitem}
\vspace{1em}

\noindent
He will thus \emph{not} exercise his option when:
\begin{compactitem}
    \item the ticket fee on the option's date of maturity is littler than the strike price ($T_m \le S$), or
    \item the customer does \emph{not} want to make use of the flight.
\end{compactitem}
\vspace{1em}

The customer's decision tree associated with this choice is provided in \autoref{fig:exercise_option}.

\insertfigure{exercise_option}{Passenger's exercise decision tree}




\section{Seller submodel}
\label{sec:sellerSubmodel}
Like the customer submodel, the seller submodel concerns itself with determining acceptable option prices. However, this model computes the prices from the perspective of the seller. This thus means that instead of calculating a willingness to \emph{pay}, the seller submodel tries to determine the best willingness to \emph{accept}. It does so by using the following equation:

$$ \mbox{WTA} = \max(0, V_s \times M) $$

Where $V_s$ is the expected standard options price according to the seller, and $M$ is the applied margin. In the benchmark configuration simulating a seller with perfect information, this margin can be seen as the part of the equation that will yield profits for the seller. The other two~cases used in this thesis --- in which the seller does not know the exact level of price movements --- the margin will also serve as a technique to reduce losses resulted by underestimation of future fare changes due to forecasting errors. The best level of this margin that will result in the optimization of total profits yielded by this simulation will be evaluated in \autoref{chap:Results}.


The first part of the equation above ensures that there is always set an option price of at least zero. This is to prevent situations from happening where the seller thinks the airfare will go down at the date of maturity, and asks a negative option price (i.e. he would offer the customer money for buying the airfare lock-in product). This is of course an undesirable effect which is averted by setting a minimum price of zero.


To calculate the standard option price $V_s$, the seller can also make use of three~different prediction methods.

\paragraph{1. Perfect foresight (used in benchmark)}
In the simulation where the seller has perfect foresight, it can $100\,\%$ accurately predict price changes of a flight. Though this is not a realistic assumption for a real-world setting, it gives valuable insights and will be used as an `optimal' benchmark to compare the other simulation models with.

\paragraph{2. Strong-form rational expectation}
The strong-form rational expectation model uses the same underlying technique as the strong-form case of the customer's prediction method. This model is being used as a `black box' to represent a relatively accurate prediction system implemented by a company. While there are too limited time and resources available to this research to actually develop such a forecasting model, research has shown that accurate models can be developed (see \autoref{sec:PredictabilityOfAirfares} for a brief literature review on this topic). In this simulation the model thus assumes such a prediction model has been developed and implemented.

\paragraph{3. Monte Carlo simulation}
The last forecasting method of the seller to determine its preferred option prices makes use of the Monte Carlo method. This is a practical forecasting model that can be easily implemented in a realistic setting. The model only uses historically available information obtained from the training set, so any predictions made in this model could also have been made when the test set was not available yet. When this model will yield positive results, it shows that even with a simplistic model the airfare~lock-in product business model could be viable.



\subsection{Seller knows exact probability of flying of customer}
Next to these different forecasting methods the seller uses, the company also has complete information on each individual customer's likelihood of travelling ($P^f$). This allows the seller to make precise estimates of the expected payout to the passengers. It will thus enable the option writer to set the option prices accordingly.

For example, when a customer has a $P^f$ of $0.70$, and the seller knows the prices will increase with US\$\,10, he can set the option price to a minimum of US\$\,7.0. This is because the company knows that \emph{on average} it only has to pay out US\$\,10 in $70\,\%$ of the cases. This is the minimum willingness to accept of the seller.

The fact that the seller is able to determine the customer's likelihood of travelling is an important concept in this model. When the option seller would not be able to do so, he would have difficulties setting prices at which the customer accepts without under- or overestimating her WTP. Determining the exact level of this parameter $P^f$ is beyond the scope of this research, but \citeA{courty2000sequential} give an explanation of how to set up a self-selecting mechanism for passengers so they reveal this information. See \autoref{subsec:AviationIndustry} for a description of this topic.



\section{General characteristics of the simulation model}
Next to the parameters defined in the two submodels, the simulation model itself also has some characteristics defined.

\subsection{Arrival rate of passengers}
To determine the performance of the model, this simulation will generate passengers. A passenger will arrive following a homogeneous Poisson distribution with an arrival rate equal to one per day for each day prior to departure for every flight:

$$ \sim \mbox{Poisson}(\lambda = 1) $$

This thus implies that on average a single customer arrives at every possibility in the simulation model.


\subsection{Number of trials}
Because pseudo random generation of numbers and probabilities are used in this model, there is a possibility of generating outliers in only a single run. To prevent such scenarios from happening, every simulation is run a number of times. The acquired data from these multiple trials is then averaged to get the converged mean.

A tests showed that the results converged to a mean at around 10~simulation runs. To be safe, the rest of the simulations will therefore use a minimum of 20~trials before evaluating the results.

Lastly, to be able to compare the models even better amongst themselves, the same random seed is set prior to running each simulation. This ensures the routes will receive the same `randomly' generated input for passenger arrival, error term, and so forth.



