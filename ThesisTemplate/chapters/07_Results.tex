\chapter{Results}
This part of the Master's Thesis will run the simulation models and compare the results of different option valuation methods. Furthermore, a sensitivity analysis will be performed to determine the influence of some of the parameters used in the models.

Due to the large amount of routes, only interesting or surprising results will be displayed and analysed in this chapter. Full tables with results can be found in the appendices \todo{ref}.


\section{Simulation 1: seller with perfect information and advance purchasing}
First, the simulation will be configuration to simulate the  seller with perfect information and the ability to purchase tickets in advance (i.e., long selling). The specific arrangement and description of its parameters can be found in \typenameref{chap:ModelDevelopment}. A summary of the level of the parameters can be found in \autoref{tbl:ParameterConfigAllKnowingSeller}.


\begin{table}
\begin{center}
\begin{tabular}{l l c r}
    \toprule
    \#  & Parameter  &  Symbol  &  Level \\
    \midrule
    1  &  option's days till maturity  &  $m$  & 3~days \\
    2  &  option's strike price  &  $p_S$  &  $p_I$  \\
    3  &  passenger's likelihood of travelling  &  $P^f$  &  U(0,1) \\
    4  &  passenger's expected increase airfare  &  $E_{p_t}$  &  6.55\,\% \\
    5  &  passenger's overestimation factor &  $\mbox{OEF}$  &  100\,\% \\
    6  &  arrival rate of passengers  &  ~  &  $(42 - m)$ per flight \\
    7  &  simulation's number of trials  &  $N$  &  150 \\
    \bottomrule
\end{tabular}
\caption{Parameters for the all knowing seller with foreknowledge}
\label{tbl:ParameterConfigAllKnowingSeller}
\end{center}
\end{table}


Next to the parameters of the simulation model, four assumptions have being made about the option seller, namely:
\begin{compactenum}
\item the seller knows the exact price movements of tickets (i.e., foreknowledge);
\item the seller knows the exact level of the customer's Willingness to Pay;
\item the seller knows in advance whether the customer will exercise its option;
\item the seller will buy tickets in advance when he knows the passenger will fly and the airfare increases.
\end{compactenum}

\vspace{1em}

The combination of these four assumptions characterize the all knowing option seller with foreknowledge. Due to these features, the option seller will be able to set the prices of options at the most optimal level. This will result in the highest level of profits a seller can ever gain from selling airfare~lock-in products in this setting.

This first simulation model will create an understanding on these maximum profits and optimal level of sales for the dataset. Furthermore, because uncertainties and risks of unavailable information have been eliminated, sensitivity analysis will also be performed. This sensitivity analysis will shed light on the influence of some of the parameters used in this section.


\subsection{Profitability of the model}
\label{subsec:ProfitabilityOfModel}
During the simulation, statistics on several properties will be collected. These properties allow the reader to get insight into the performance of the models. However, due to differences amongst the routes in terms of number of collected flights and height of ticket prices, comparison of performance between different outbound/inbound pairs is more difficult. Therefore, a ratio is introduced in this research, which enables better analysis between models and flight. The value is computed by multiplying the mean relative profits per option by the total percentage accepted, or:
$$\Pi_M = \Pi_O \times q$$

Where $\Pi_M$ is the profitability of the model, $\Pi_O$ denotes the profitability of an option, and $q$ represents the total number of options accepted by the passengers. The former two of these variables are in percentages relative to the mean ticket price, $\mu_{p_t}$, while the number of options accepted is the percentage of total generated customers.

This metric thus shows the expected profits as a percentage of the ticket price for every customer that requests an offer for an option. For example, consider a simulation model with 10,000 generated passengers. Of these customer's, 5,000 retrieve an offer at a level that they accept. After completion of the run, the model calculates that on average an accepted option yields 2\,\% of the ticket price as a profit. The profitability of the model thus is $0.02 \times 0.50 = 1\,\%$ profits per incoming customer. One can interpreted this as that for every possible customer that arrives in the model, the option seller is expected to gain 1\,\% of the ticket price in profits.

Because the measurement consists of only relative numbers, it can be used to compare outcomes of flights with a different number of generated passengers or ticket prices. This variable will thus be used throughout the rest of the model to compare distinct routes with each other.


\subsection{Analysis of the results}
\label{subsec:AnalysisPerfectInformation}
The simulation was first run on all the different routes using the parameters defined in previous section. The aggregated results for all routes can be found in \autoref{tbl:resultsAllKnowing}.


\begin{table}
\begin{center}
\begin{tabular}{c c | c c | c | c }
\toprule
    \multicolumn{2}{c|}{$\mu_{p_O}$} & \multicolumn{2}{c|}{$\Pi_O$}  &  $q$  & $\Pi_M$ \\[.4ex]
    \$  & \%  &  \$  & \%  & \%  & \% \\
    \midrule
16.50  &    3.28  &   16.50   &    3.28  &   100  &   3.28 \\
    \bottomrule
\end{tabular}
\caption{Simulation results of all knowing seller}
\label{tbl:resultsAllKnowing}
\end{center}
\end{table}


The columns labelled with $\mu_{p_O}$ contain the mean price of the option. The first column denoted the numeric value, while the second one gives the number in percentages relative to the mean ticket price. $\Pi_O$ denotes the profits gained per accepted option, and $q$ the percentage of options accepted. Lastly, $\Pi_M$ is the profitability of the model as described in \autoref{subsec:ProfitabilityOfModel}.

The outcomes of the model are in alignment with the expectations. The optimal option price on average is 3.28\,\% of the current ticket price, which is exactly half of the expected increase according to the passenger (6.55\,\%). This can be explained by the fact that the customer is willing to pay this expected increase multiplied by his likelihood of flying, $P^f$. Because the probability of flying is randomly drawn from an uniform distribution, the average of $P^f$ is $0.5$. Hence the average option price of 3.38\,\%. This was the case for every individual route in the simulation.

As also is shown in the table, the average profits gained for every accepted option -- $\Pi_O$ -- are equal to the average option price $\mu_{p_O}$. This is because of the fact that the seller has perfect information and is allowed to purchase tickets in advance. In this simulation, the option writer can make sure to buy the tickets at no price higher than the strike price, $p_S$. This is illustrated by the option seller's decision tree in \autoref{fig:PerfectInformationTree}. Because the seller has perfect information and makes his forecasts with 100~percent accuracy, he can purchase the tickets at the right time.

In the first case where the seller knows the passenger will never exercise the option, he will not undertake any action. The price the customer paid for the airfare~lock-in product is in its entirety profits to the seller.

The other cases represent passengers that will exercise their options on the date of maturity. When the price of the underlying ticket does \emph{not} increase, the seller is able to wait with the purchasing of the flight until maturity. However, when the airfare \emph{does} increase, the writer will be able to purchase the ticket in advance at the lower rate. Because of his foreknowledge, he does indeed have the information to choose the best time for buying a flight.

\insertfigure{PerfectInformationTree}{Decision tree of the all knowing seller with foreknowledge}

Because the option seller is always able to buy tickets for a flight at the lowest possible costs (i.e., at no level higher than $p_S$), he is also able to always offer an option to every passenger that has a request. This is represented by the $q$ of 100~percent in the table.

The profitability of the model in this case is also 3.28\,\%, the same as the average percentage of profits per airfare~lock-in product. This is explained by the $q$ of 100~percent. The profitability of the model thus shows that for every passenger that requests an option price in the simulation, the option seller is expected to gain 3.28\,\% of the ticket price in profits. This is the absolute maximum case in this configuration and will thus be used to compare other models with. 


\subsection{Sensitivity analysis}
After the first simulation, a sensitivity analysis was performed. The influence of the variables that were used to calculate the customer's Willingness To Pay were altered to determine their influence on the outcomes of the model.

The results of the effect of changes in the parameters were as could be theoretically expected. When a customer overestimated the expected increase in airfare by a single percent point, his Willingness To Pay increased with half a percent point. When the $\mbox{OEF}$ of a passenger decreased by a certain amount, the WTP would also decrease by half of that amount. 

$$ \Delta{\mbox{OEF}} = 0.5 \times \Delta{\mbox{WTP}} $$

This linear relation is easily explained by the fact that the customer's Willingness To Pay is closely related to his $P^f$. Because the average probability of flying is 0.5, every percentage increase results in a half percent increase of the WTP. Because in this first simulation it is assumed that the seller knows the exact level of the customer's WTP, the selling price of the airfare~lock-in product will automatically increase.

The same effect can be seen when increasing the number of days till maturity. After running the simulation, all of the expected outcomes were half of the passenger's expected increase in airfares.


\section{Simulation 2: seller with perfect information but without advance purchasing}
The second simulation will be based upon the previous run, but drops a single assumption. This model will simulate an option seller with perfect information that does not have the possibility to purchase tickets in advance. He therefore will not be able to always purchase the tickets at the lowest price, and has to wait till maturity before he can buy the flight ticket when requested by the customer. However, this model still assumes that the option seller has the ability to make predictions with a 100~percent accuracy, and also does know whether the passenger will exercise his option on the date of maturity. The decision tree in \autoref{fig:NoAdvanceSellingTree} gives an overview of the decision process of the option seller that has no ability of buying the tickets in advance.

\insertfigure{NoAdvanceSellingTree}{Decision tree of the all knowing seller with foreknowledge but no advance purchase}
 
As can be observed from the figure, the option seller will only offer an option product to the passenger when he either knows the customer will not exercise the option, or when the future ticket price is lower than the strike price plus the Willingness To Pay of the consumer. Because in this simulation the strike price, $p_S$ is always equal to the initially observed airfare, $p_I$, the latter case can also be described as when the price increase of an option is less than the price of the customers WTP.




\begin{table}
\begin{center}
\begin{tabular}{c c | c c | c c | c | c }
\toprule
\multicolumn{2}{c|}{Airport}  &  \multicolumn{2}{c|}{$\mu_{p_O}$} & \multicolumn{2}{c|}{$\Pi_O$}  &  $q$  & $\Pi_M$ \\[.4ex]
from  &  to  &  \$  & \%  &  \$  & \%  & \%  & \% \\
    \midrule
AMS  &  CDG  &    6.48  &  2.97  &    6.16  &  2.83  &  76.59  &  2.17 \\
~    &  DXB  &   18.38  &  3.16  &   18.24  &  3.14  &  90.11  &  2.83 \\
~    &  JFK  &   25.61  &  3.11  &   23.81  &  2.88  &  83.90  &  2.41 \\
~    &  LHR  &    7.93  &  3.07  &    7.36  &  2.84  &  81.05  &  2.31 \\[.5ex]
ATL  &  LAX  &   11.58  &  3.06  &   10.57  &  2.78  &  80.13  &  2.23 \\
~    &  MCO  &    7.23  &  2.96  &    6.81  &  2.80  &  75.90  &  2.12 \\[.5ex]
CDG  &  AMS  &    6.90  &  2.99  &    6.58  &  2.86  &  77.59  &  2.22 \\
~    &  LHR  &    6.68  &  3.08  &    6.26  &  2.88  &  82.17  &  2.37 \\[.5ex]
DEN  &  FLL  &    5.61  &  3.12  &    5.29  &  2.96  &  84.39  &  2.50 \\
~    &  PHX  &    5.55  &  3.08  &    5.09  &  2.82  &  81.40  &  2.29 \\[.5ex]
DFW  &  LAX  &    8.67  &  3.03  &    8.13  &  2.84  &  79.82  &  2.27 \\
~    &  SFO  &   11.32  &  3.02  &   10.59  &  2.82  &  78.67  &  2.21 \\[.5ex]
JFK  &  CDG  &   89.64  &  3.16  &   86.05  &  2.96  &  87.94  &  2.61 \\
~    &  LHR  &   30.24  &  3.17  &   27.97  &  2.90  &  87.82  &  2.55 \\[.5ex]
LAX  &  LAS  &    6.48  &  3.08  &    6.08  &  2.88  &  82.59  &  2.38 \\
~    &  SFO  &    6.81  &  3.03  &    6.39  &  2.84  &  79.39  &  2.26 \\[.5ex]
LHR  &  AMS  &    7.32  &  3.06  &    6.71  &  2.80  &  80.22  &  2.25 \\
~    &  JFK  &   27.79  &  3.23  &   25.51  &  2.93  &  91.94  &  2.69 \\
~    &  LAX  &   41.55  &  3.22  &   38.19  &  2.90  &  90.62  &  2.63 \\
~    &  MAD  &    8.37  &  3.07  &    7.82  &  2.87  &  82.03  &  2.36 \\[.5ex]
ORD  &  BOS  &    7.46  &  3.06  &    6.80  &  2.77  &  79.90  &  2.22 \\
~    &  LGA  &    8.33  &  3.12  &    7.73  &  2.88  &  84.19  &  2.43 \\
\midrule
Means &  ~   &   16.18  &  3.08  &   15.19  &  2.87  &  82.65  &  2.38 \\
    \bottomrule
\end{tabular}
\caption{Simulation results of all knowing seller without advance purchasing}
\label{tbl:resultsSecond}
\end{center}
\end{table}


