\chapter{Research Questions}
\label{chap:ResearchQuestions}
In this thesis, I would like to theoretically and empirically evaluate whether it is viable for an external company to offer airfare~lock-in~products. In my research, I refer to \emph{viable} as the availability of an option price that both satisfies the option buyer (i.e., customer) and the option writer (i.e., the external company). The external company's minimum price at which he is still willing to sell the option (i.e., \emph{willingness to accept} or \emph{WTA}) is equal to the expected costs of the risk associated with this product. This risk can be expressed as the expected increase in ticket price combined with the probability of the costumer not exercising the option\footnote{This only applies in a risk-neutral, fully predictable setting with rational customers that have equal information. See \autoref{sec:Passenger} for a more detailed description of the theoretical scenario's}. The customer will only accept the offer when his willingness to pay for the option (i.e., \emph{WTP}) is higher than this minimal price. In this thesis I will therefore test what the minimum price of the offer should be according to the seller, and compare this with the maximum price that the customer's \emph{WTP}. The main research question in this paper can be defined as:

\begin{quote}\emph{Is it sustainable for an external company without capacity information or seat reservation capabilities to set the price of airfare~lock-in~products at a level that the customer accepts?}\end{quote}

This research question will thus engage itself in determining whether a third party could offer options on ticket prices. The company should in such situations be able to generate revenues and profits by selling these options to passengers. It can do so by making sure it sets --- on average --- the right prices for these lock-in product. A number of seller variables are related to this decision.

\paragraph{Forecasting method} This variable represents the method which the \emph{seller} uses to make future price predictions. This thesis will compare the use of three~different prediction techniques:
\begin{compactitem}
    \item perfect predictions,
    \item strong-form rational expectations, and
    \item using Monte~Carlo simulation.
\end{compactitem}

A certain level of accuracy is associated with the different forecasting methods the writer uses. The first two methods will base their predictions on the actual ticket pricing data that are being used to test the simulation models. The last forecasting strategy will use historical available information to make its predictions.

\paragraph{Applied margin} The seller sets his willingness to accept to the expected pay of writing the option. In this setting, the seller will be unable to make any profits as --- on average --- he will gain as much revenues as he has to pay out to the customer. Furthermore, due to the inaccuracies described in the previous item, the errors will yield negative results as the seller would sometimes accidentally offer options under their cost price. To compensate for these two factors, the company can set a certain level of \emph{margin}, which allow him to set prices above expected costs. This variable is thus related to the total profits an airfare~lock-in product seller will gain.

The second research question will concern itself with these previously described parameters, and will be defined as:

\begin{quote}\emph{How can an airfare~lock-in product seller make use of his forecasting method and applied margin to optimize total profits?}\end{quote}

Like the option seller, the customer also uses a certain prediction technique to determine his maximum level he is willing to pay for the option. This research considers two~types of customer's forecasting methods:
\begin{compactitem}
    \item strong-form rational expectations, and
    \item weak-form rational expectations.
\end{compactitem}

The third and last research question in this thesis will consider the impact of the choice of prediction technique, and can be defined as:

\begin{quote}\emph{What is the influence of the customer's forecasting technique on the total profits of the airfare~lock-in~product seller?}\end{quote}

To determine the answers to these research questions, a number of simulations will be run. The outcomes of these simulation models will be used to compare different configurations of sellers and customers.
