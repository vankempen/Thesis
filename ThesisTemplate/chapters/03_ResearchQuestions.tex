\chapter{Research Questions}
\label{chap:ResearchQuestions}
In this thesis, I will theoretically and empirically evaluate whether it is viable for an external company to offer airfare~lock-in~products. In my research, I refer to \emph{viable} as the availability of an option price that both satisfies the option buyer (i.e., customer) and the option writer (i.e., the external company). The external company's minimum price at which he is still willing to sell the option (i.e., \emph{willingness to accept} or \emph{WTA}) is equal to the expected costs of the risk associated with this product. This risk can be expressed as the expected increase in ticket price combined with the probability of the costumer not exercising the option\footnote{This only applies in a risk-neutral, fully predictable setting with rational customers that have equal information}. The customer will only acceot the offer when his willingness to pay for the option (i.e., \emph{WTP}) is higher than this minimal price. I will therefore test what the minimum price of the offer should be according to the seller, and compare this with the maximum price that the customer's WTP.

In my research, I will consider an airfare~lock-in product provider that has no extra capabilities other than publicly available. The model therefore assumes that the seller has no information on current capacity load of the underlying flight, nor that he has he the possibility to reserve this capacity. The main research question can therefore be defined as:

\begin{quote}\emph{Is it sustainable for an external company without capacity information or seat reservation capabilities to set the price of airfare~lock-in~products at a level that the customer accepts?}\end{quote}

A company might decide to enter the lock-in products market when this research question yields positive results. The customer, on the other hand, will only purchase such products when she will gain something as well. The second research question will deal with this inquiry, and concerns itself with the results from the passenger's perspective. It will do so by trying to determine whether or not it is beneficial to a passenger to purchase an option on the underlying flight.

\begin{quote}\emph{What are the implications for the customer resulted from the purchase of an an airfare~lock-in product?}\end{quote}

As stated previously, the first research question will engage itself in determining whether a third party could offer options on ticket prices. The company should in such situations be able to generate revenues and profits by selling these options to passengers. It can do so by making sure it sets --- on average --- the right prices for these lock-in product. A number of seller variables are related to this decision.

\paragraph{Forecasting method} This variable represents the method which the \emph{seller} uses to make future price predictions. This thesis will compare the use of three~different prediction techniques:
\begin{compactitem}
    \item perfect predictions,
    \item strong-form rational expectations, and
    \item using Monte~Carlo simulation.
\end{compactitem}

A certain level of accuracy is associated with the different forecasting methods the writer uses. The first two methods will base their predictions on the actual ticket pricing data that are being used to test the simulation models. The last forecasting strategy will use historical available information to make its predictions.

\paragraph{Applied margin} The seller sets his willingness to accept to the expected pay of writing the option. In this setting, the seller will be unable to make any profits as --- on average --- he will gain as much revenues as he has to pay out to the customer. Furthermore, due to the inaccuracies described in the previous item, the errors will yield negative results as the seller would sometimes accidentally offer options under their cost price. To compensate for these two factors, the company can set a certain level of \emph{margin}, which allow him to set prices above expected costs. This variable is thus related to the total profits an airfare~lock-in product seller will gain.


Like the option seller, the customer also uses a certain prediction technique to determine his maximum level he is willing to pay for the option. This research considers two~types of customer's forecasting methods:
\begin{compactitem}
    \item strong-form rational expectations, and
    \item weak-form rational expectations.
\end{compactitem}

\noindent
The third and final research question in this thesis will consider the , and can be defined as:

\begin{quote}\emph{What are the effects of the option seller's level of forecasting accuracy and applied margin, and the customer's level of accuracy on the profits gained when selling airfare~lock-in products?}\end{quote}

The rest of this thesis will use simulation models and empirical analysis to provide answers to these research questions.