\chapter{Literature Review}
\label{chap:LiteratureReview}
\section{Yield management in the aviation industry}
According to \citeA[pp. 2]{anderson2004revenue} the term yield management can be described as ``\textit{[the] process of managing perishable inventories to maximize the total revenue from these inventories}''. The aviation sector was the first industry to apply this concept of revenue management. The costs of the services airlines sells are mostly fixed (e.g., fuel, maintenance) and consist  of negligible variable costs (e.g., meals) \cite{o2001introduction}. In such a scenario, unsold seats become obsolete upon departure and would result in lost revenues to the company. A more airline-related definition of yield management is given by \citeA[pp. 136]{akgunduz2007using}, who state that ``\textit{revenue management is [\ldots] the practice of managing the booking requests with an objective of increasing the sales revenues}''.

\citeA{smith1992yield} defined three major changes in the aviation industry which contributed to the development of yield management. These changes, which all occurred in the 1960's and 1970's, had to do with:

\begin{description}
\item[Deregulation]
Arguably the most important factor in the creation of revenue management in the aviation industry is its deregulation. The deregulation was formed by the approval of the \citeA{airlinederegact}, and removed price, route and entry restrictions on airlines. After removal of these restrictions, airliners were allowed to set their own prices. The ability to change prices for different segments of customers is one of the core concepts of revenue management.

\item[Automated reservation systems]
Computerized reservation systems allowed airlines to centrally control reservation activity, and process and store huge amounts of data. With the development and implementation of the Semi-Automated Business Research Environment (SABRE) in 1966, American Airlines pioneered in the field of online reservation systems for the travel industry \cite{voneche2005yield}. The data acquired via the system could be analysed and used to forecast the booking process, and automatically adjust airfares of flights according to their demand.

\item[Discount fares]
The idea behind the concept of discount fares is to offer reduced prices to specific segments of customers to sell seats which would otherwise be empty \cite{belobaba1987survey}. This concept added a whole new dimension to the airline industry. For instance, by offering low prices to low-grossing customers early in the booking process, there would be fewer seats to sell to high-grossing travellers. The concept of determining the right amount of tickets with discount at specific times is called \emph{Discount allocation}.
\end{description}

Discount allocation has its origin in the article by \citeA{littlewood1972forecasting}. In his paper, for which he would receive the \citeA{informs13} in 2004, Littlewood described a number of approaches to handle the allocation of discount fares in the airline industry. By predicting future demand and analysing consumer sell-up behaviour, one would be able to create a better allocation of price discounts which resulted in a higher load factor and revenues for the airlines. More specifically, Littlewood's rule states that a ``\textit{request for a seat should be fulfilled only if its revenue exceeds the expected future value of the seat in question}'' \cite[pp. 17]{wright2010dynamic}.

For the concept of discount allocation, historical pricing and demand acquired by the implementation of computerized reservation systems proved to be very useful to create more accurate predictions \cite{littlewood1972forecasting}.

In their paper, \citeA{smith1992yield} describe the case of revenue management at American Airlines. Of all the airliners, American Airlines was the first to actually implement the theory of yield management in the 1970s. After years of research in the 1960s, the company developed an application using the data acquired via their computerized reservation system, SABRE. The system, which was later named DINAMO, was based upon the concepts described by \citeA{littlewood1972forecasting}, and implemented the options of overbooking, discount allocation and traffic management \footnote{traffic management is the optimal reservation of single-legged flights to provide for higher-profit multi-legged flights, while preventing empty seats}. The system decreased the spoilage of unsold seats from 15~percent to 3~percent \cite{smith1992yield}. Furthermore, in 1992, the former CEO of American Airlines, Robert Crandall, showed that yield management contributed to an increase of \$~1.4 billion in revenues and an estimated \$~500 million of extra profits in the first three years after implementation of DINAMO.

While the introduction of discount fares increased utilization of aircrafts and significant expanded revenues of airliners, it also resulted in higher uncertainty for consumers. Where the customers first could base their decisions on linear prices, airfares have become very dynamic and follow unintuitive paths. As an example, \citeA{groves2013agent} state in their paper that, in contrast to popular believe, the principle to buy tickets early in the booking process to avoid an increase in price is suboptimal. Their research shows that this strategy almost always results in higher prices and could be easily outperformed by other models (see \label{subsubsec:PredictabilityOfAirfares} for a complete description of their article).

To overcome the problem of uncertainty created by yield management, new theories of pricing have been developed. One such example is the theory of \emph{real options} as described by \citeA{amram1999real}. These specific kind of options use the framework developed by modern financial instruments to cope with uncertainty of underlying assets. In accordance, \citeA{dixit1996investment} argue that in the presence of uncertainty and managerial flexibility in operating strategies, real option may provide a solution.

The practice of yield management is not only restricted to the aviation industry, but applies to many other sectors. \citeA{kimes1989yield} supplies a list containing common characteristics of industries that successfully applied revenue management:
\begin{compactitem}
\item relatively fixed capacity;
\item ability to segment markets;
\item perishable inventory;
\item products sold in advance;
\item fluctuating demand;
\item low marginal sales cost/high marginal capacity change cost.
\vspace{1ex}
\end{compactitem}

\citeA{harris1995hold} give in their paper a few examples of industries that all share these characteristics, namely: hotels, car rental agencies, railroads, cruise ships, printing and publishing. Dynamic pricing due to revenue management practices creates uncertainty on the customer-side in these sectors as well. So, the theory of real options could also be applied to these industries.

In the next section I will analyse current research on this topic and give an over\-view of option-based models applied to these kind of industries.

\section{Real options in current industries}
An option gives its holder the right, not the obligation, to buy or sell (respectively named \emph{call}- and \emph{put}-options) a specific underlying asset on or before a certain date (\emph{maturity}) for a predetermined price (\emph{strike price}) \cite{hull99}. A real option differs from a financial option, in that the underlying asset is a tangible object, rather than a derivative. Furthermore, real options are not normally not tradable like financial options are. Therefore they are often referred to as \emph{deferral options} \cite{jain11}.

A real option is most commonly used as an alternative to the \emph{net present value}-method for managing capital budgeting. However, this type of could also be applied to set up contracts for other tangible assets. This last category of options can be offered to consumers to sell for example tickets for games, energy or insurance. In their paper, \citeA{sainam2010consumer} provide another type of option even more consumer related. The option, for which they coined the term \emph{consumer option}, is like a real option, but the quality of the outcome is different per consumer segment. The authors give the example of the final of a sports tournament, where a customer only wants to see the final when his team is playing. When the customer's team did not make the finals, he or she will not exercise the option.

The term \emph{consumer option} is not often used in literature, and authors usually use \emph{real options} to also refer to consumer options. To avoid confusion, I will use the term \emph{real option} to refer to all options with tangible underlying assets which are provided to consumers.

An important distinction between options is the time at which holders are allowed to execute the contract \cite{hull99}. On the one extreme, there are \emph{European} options, which allow the customer to only execute its option \emph{on} the specified maturity date. At the other end of the spectrum are \emph{American} options, which allow the holder to execute their right \emph{before} the determined maturity date. In between these two types of options there are many alternatives, like Bermudan or switch options which allow execution of options a certain number of times before maturity date. An example implementation of such an option can be found in utility markets.

\subsection{Utility industry}
An industry in which a real-option model is already applied, is the utility industry \cite{keppo2004pricing}. The options, called \emph{swing options}, are traded over-the-counter by energy suppliers to their customers. A consumer buys a forward contract which obliges him to buy a predetermined amount of electricity for a certain period at a specific price. To allow for flexibility, a customer can bundle this forward contract together with the swing part, which gives him the right to change the amount a number of times during this period \cite{jaillet2004valuation}. The swing option thus gives the holder the opportunity to have a set price, while he is still able to respond to uncertainty by changing the amount of energy when circumstances change.

\citeA{thompson1995valuation} was the first researcher to determine the optimal price of such a swing option. He created a tree-based path-dependent  model based upon previous research by \citeA{hull1990valuing}, and used dynamic programming to numerically approximate the best value of these contracts. The author admits that solving path-dependent is difficult, because - in the case of swing options - prior exercise decisions influence the outcome of the model. This makes it very hard to determine the optimal price of the swing option analytically.

\citeA{jaillet2004valuation} present in their paper a valuation framework for swing options. The authors based their model on multilayered trinomal trees. The concept of trinominal trees for the use of option valuation was first used by \citeA{boyle1986option}, and allows to evaluate the contract on multiple decision variables. \citeA{jaillet2004valuation} also state in their paper that exact valuation of options in path-dependent models is difficult, and their proposed model faces the same difficulties as in other real option frameworks.

Both of the previously described techniques use numerical techniques that use trees to evaluate the optimal price of a swing contract as a whole. \cite{keppo2004pricing} on the other hand, replicates the swing contact by a future contract and a call option. The author estimates price the call part by looking at the commodity exchange market by using the Black--Scholes method \cite{black1976pricing}. He also concludes that due to the swing option's path dependency and Bermudan style, it is not possible to solve the problem algebraically.

Swing options have some similarities with airfare lock-in products, but there are also large differences. For example, costs when switching from energy providers are high and often impossible due to contracts. Customers are `stuck' at their current energy provider, while in the airline industry, one can easily switch between airliners without additional costs. Furthermore, a swing option can be seen as multiple European and/or American options, which creates path dependency. Options for airline tickets consist of a single European or American option. Lastly, \citeA{jaillet2004valuation} state that the uncertainty of a swing option is determined by the unpredictability of prices (future part) combined with the quantity which is actually needed (swing part). The uncertainty of airline tickets, however, is primarily based upon the unpredictability of prices.

\subsection{Car rental industry}
Car rental agencies continuously face the problem of allowing a vehicle to be booked for a future time right now, or rather wait and sell the car at a higher rate in the future. When the agency decides to rent out the car right now there might be missed revenues due to not being able to offer high-end customers a product. On the other hand, when agencies refuse the current booking, they face the probability that the car is not booked at all. This would result in even higher loss of revenues as the car has `perished' and can never be rented out for that passed date again. According to \citeA{anderson2004revenue}, this classic revenue management problem is similar to the swing option method described in previous section. They present the case of real options applied to the car rental industry by trying to apply the swing option model currently available in electricity and gas industries to the operating agent of such a company. In the paper the authors were able to solve the operator's optimal rental strategy by considering the operator as the holder of a swing option on car rentals.

While \citeA{anderson2004revenue} state in their article to have found an optimal solution to this problem which is extendible to other industries, they also indicate some difference among the car rental and airline industry.

First, the authors state that airliners offer airfares by opening and closing discrete pricing classes. While the car rental industry also has different classes (size, AC, \ldots), they also differ prices within each class. This form of dynamic pricing is more difficult to predict, and hence creates more uncertainty. Secondly, car rental customers are less prone to changes in fares, while this is not the case in the airliner industry. And lastly, this model might not be applicable to this research as it defines the concept the other way around. The framework presents the idea in which the operators use swing-like options to determine the best car rental strategy. However, this paper focusses on the ability of \emph{customers} to buy real options to lock in airfares.

\subsection{Tournament industry}
A more recent adoption of real options handling uncertainty, is the category associated with the sports market. Tickets for games usually sell weeks before the actual game. Fans of a team are willing to pay a large sum for a ticket when their team actually plays in the finals, but less or even nothing when this is not the case. However, the outcome of the actual teams that play in the finals is often not known in advance. Furthermore, customers need to buy tickets early in advance to prevent sell-outs. This puts the fans at risk, because there is a probability that their favourite team does not play in the finals. According to \citeA{sainam2010consumer}, this uncertainty creates a suboptimal situation where ticket agencies cannot charge the best prices due to customers taking into account the possibility that they buy a ticket for a final in which their favourite team does not play. The authors propose a new concept in which, instead of the ticket, customers buy real options. This real option gives its holder the right to buy the actual ticket when the outcomes are known.

To test this concept, \citeA{sainam2010consumer} empirically compare three pricing mechanisms, knowing:
\begin{description}
\item[Selling in advance] This is the current situation. Customers buy tickets well in advance and hope that their team plays the actual finals;
\item[Pricing on full information] Customers are sold tickets after the outcome of teams in the final is known;
\item[Providing real options] In this concept the customer first buys an option well in advance which gives him the right to attend the game. When the outcome is known, the customer can decide whether to exercise the option for an extra amount.
\end{description}

The authors studied the concept using these three pricing mechanisms, and showed that in many cases revenues where higher when applying the \emph{real options}-model, rather than \emph{advance} or \emph{full information} selling.  This was due to the fact that agencies can charge higher prices, because the \emph{willingness to pay} of customers is not driven down by uncertainty. The concept also generates extra revenue from selling real options, and expands the existing market due to customers willing to buy tickets at a low rate of uncertainty. The profit advantage for option selling was highest when uncertainty was highest, which clarifies the purpose of options. The paper also found that \emph{risk-averse} customers are willing to pay more for their ticket under uncertainty than their \emph{non-risk averse} counterpart.

\citeA{berkowitz2012real} even go further into this concept and propose a concept in which customers can \emph{only} exercise their options when their favourite team reach\-es a given round. Furthermore, the authors suggest that a combination of advanced selling and real options might increase a ticket agency's revenue even further due to segmentation of game- and team-based fans. In their paper, the authors test this proposition and come to the conclusion that not only the revenue increases for the organizer, but also the utility of fans due to the fact that they are sure to have a ticket when their team plays in the finals. Lastly, more fans from the playing teams are present, which could increase the experience of the spectators.

\citeA{balseiro2011revenue}, who apply a combination of the models proposed by \citeA{berkowitz2012real} and \citeA{sainam2010consumer}, also show these increases in revenue. They argue that this is due to the ability to better segment customers and thus better determine their \emph{willingness to pay}. Furthermore, they argue that the extra certainty increased social welfare of the customer.

\citeA{berkowitz2012real} state in their paper that their model could be used in many other industries as well. The authors give examples of which airline tickets is amongst them. While similarities can be found on a very abstract level (like the \emph{binary option approach}), there seem to be major differences among the industries. For example, finals for sport events have close to no fluctuations in price and a very high sell-out probability closer to the date of the final. In the aviation industry, however, high fluctuation in price is much more common. Furthermore, flights do sell-out, but often this is not the case \footnote{currently the average load factor is 83.6~percent. See \url{http://www.transtats.bts.gov/}}. Also, the few number of alternatives in sporting events creates less price elasticity of customers. This is not applied to the aviation sector where a small change in price might cause a customer to switch to another vendor.

\subsection{Aviation industry}
\label{subsec:AviationIndustry}
The first occurrence of option pricing in the aviation industry in literature is provided by \citeA{walker1998method}. In their patent, the authors provide a system which determines the price of an option on a specific flight. This option gives the holder the right to buy the ticket at a moment in the future at the specified price, without actually holding the risk of sell-outs and increases in airfares. According to the authors, real options sold in this kind of setting have three major differences compared to traditional financial options:
\begin{compactenum}
\item the product is only supplied by a number of airlines;
\item the product is not interchangeable, because customer might only want to fly with a specific airline at a specific time and date;
\item there are only a finite number of seats available (capacity constrained).
\vskip 1ex
\end{compactenum}

To overcome these differences, they describe a system which evaluates options with a method other than the traditional systems. The method determines the price of an option by multiplying a base price for the option by several relevant factors. These factors include: \emph{number of days before departure}, \emph{expected load of aircraft}, \emph{desirability of the customer} (low-grossing versus high-grossing), \emph{flexibility of the customer's flight plans}, and \emph{historic volatility of ticket prices}. 

\citeA{jain11} argue in their paper that traditional option valuation methods like the \emph{Black--Scholes} model \cite{black1976pricing} can actually be applied to this type of real options. The followed prices of 14~flights for 79~days during a period from November~2005 till February~2006. After collection of the data, the authors used the Black--Scholes model to evaluate the theoretically best price for a call- and put-option with a maturity date of 79~days (i.e., ticket bought on 79~day before departure) on these specific flights. The authors calculated the affordability of each option, which they defined as price of an option divided by the average ticket price of the flight ($p_o/\bar{p_f}$). Their results show that the affordability of lock-in~products on their analysed flight range from 2.46~percent to 11.28~percent. The average affordability across all 14~flights is 7.2~percent. Put options to lock-in the price range from 1.55 to 10.34 with a mean of 6.37~percent of the ticket's price.

One major limitation according to \citeA{jain11}, is that their analysis is based on \emph{ex post data}. This means that the option price, including the volatility, was based upon data that was only available to the customer in the future. While the authors recognize this as a major flaw in their research, they argue that this data is available at the airliners.

Other types of real options are more prevalent in the aviation industry. For example, \citeA{akgunduz2007using}, \citeA{gallego2008revenue}, \citeA{ching2010improving}, all describe the use of options in the booking process of an airliner. They illustrate the concept in which airliners sell tickets with a very high discount to customers early in the booking process. Coupled with these low-fare tickets the airliner buys a call option, which gives them the right to recall the ticket when capacity is needed for higher-grossing customers. This gives the airliner the opportunity to sell tickets at high-fares when demand is high, while still having the backup of the low-fare customer to prevent empty seats. A put option is also described, in which a travel agent agrees upon buying tickets from the airline at a very low-fare when these are still available close before departure of the flight. This allows the airliner to still sell tickets which would have been empty seats otherwise.

The previously mentioned authors all use numerical methods to determine the optimal mix of call/put and normal seats. Their shared conclusion is that this type of yield management is superior to the current overbooking methods, because it generates more revenues and also higher social welfare for customers. The model is also applicable to other concepts. \citeA{graf2011option} and \citeA{graf2013transfer} show for example that this concept of capacity based option pricing can be implemented at airline alliances. In this situation, the operating airline can sell options to the ticketing airline. When the ticketing party sells a seat of higher value than the strike price of the option, it exercises the option. The operating carrier can also buy a put option on the sold call option. These kind of option on options, also defined as \emph{compound options} \cite{trigeorgis1996real}, give the operator the right to recall options from the ticketing company when demand is high. In a strategic alliance this only happens when the revenue gained from accepting a seat request is greater than the strike price plus the option price \citeA{graf2011option}. The authors use simulation-based evaluation of booking limits (i.e., how many calls and puts to sell) to determine the best scenario of such an options.

\section{Option valuation methods}
\subsection{Theoretical model: Black--Scholes}
The most used options valuation method for pricing options in financial markets is the Black--Scholes model \cite{hull99}.  It was first proposed in a paper by \citeA{black1973pricing}, and was later expanded mathematically by \citeA{merton1973theory}. The model calculates the value of an option for the underlying stock. The article shows that the expected rate of return of the underlying asset and risk preference of the investor do not matter. The formula rather uses other variables (e.g., volatility of the underlying asset, maturity, risk-return rate) to estimate the theoretical option price. The Black--Scholes formula is defined as follows:
\begin{align*}
C(S, t) &= N(d_1)S - N(d_2) Ke^{-r(T)} \\
d_1 &= \frac{\ln(S/K) + (r + \sigma^2/2)(T - t)}{\sigma \sqrt{T - t}} \\
d_2 &= d_1 - \sigma \sqrt{T - t}
\end{align*}

\noindent
\citeA{black1973pricing} make a few assumptions of the underlying stock, including:
\begin{compactitem}
\item The option style is European;
\item The stock price follows a geometric Brownian motion (i.e., log-normal distribution) with constant drift and volatility;
\item The stock pays no intermediary dividends.
\end{compactitem}

\noindent
These assumptions might also be applicable to airfares. For instance, a rational customer would only exercise its airfare~lock-in product at maturity (see \typenameref{sec:OptionStyle} in \cref{chap:Methodology} for a detailed explanation). Furthermore, the numerous factors and entities that influence the price of a ticket can be very unpredictable. This implies a random walk where price increases and decreases follow a certain log-normal distribution with an upward trend.

The model proved itself to be very useful on the stock-market and was quickly implemented. While implementers of the tool use it often with adjustments, the equation approaches the actual observed option prices on the exchanges \cite{bodie2008investments}. 
Next to option valuation in financial markets, the Black--Scholes-model proved itself to also be applicable to other assets. For example, \citeA{finch1998option} give the example of the valuation of the options held by a firm which allows it to abandon the capital investment of a project prior to completion. Another more recent --- but theoretical --- implementation of the method is shown by \citeA{jain11}, who showed that the theory might also be applicable in the airline industry. A full description of this paper can be found in \typenameref{subsec:AviationIndustry}.


\subsection{Numerical approximation: Monte Carlo}
Rather than an analytical approach, the \emph{Monte Carlo} method uses simulation to approximate the outcome of a model. It does so by generating large quantities of pseudo-random numbers that follow certain distributions, and use these as input for the variables used in the model.  By running the model many times, the outcomes of the model will converge to a particular result. This convergence is caused by the \emph{law of large numbers}. That the simulation method can be quite accurate after a certain amount of trials is shown by \citeA{richardson2009numerical}. In his article, the author compared option prices yielded by the theoretical Black--Scholes model with the ones produced by a Monte~Carlo model. Both approaches assumed a distribution with log-normal returns. He showed that after a small number of trials (i.e., 5\,000), the Monte Carlo method differed from the theoretical model by a few thousands of a Dollar.

The first to use Monte Carlo model for the valuation of financial options was \citeA{boyle1977options}. In his paper, the author shows that Monte Carlo simulation can be used as a different approach to option valuation. For instance, the author gives the example of a model in which also the price of European options on dividend-paying derivatives can be estimated.

The approach of simulation in option valuation techniques thus has a few advantages compared to more exact methods like Black--Scholes. \citeA{boyle1977options}, for example, state that Monte Carlo can make use of different distributions. As described in previous section, the Black--Scholes formula assumes a geometric Brownian motion. When the underlying asset does not follow such a log-normal distribution, this method will yield unreliable results. A Monte Carlo model, however, can be used to exactly represent the probability density function of underlying returns. This method might therefore be more applicable, as the \typenameref{sec:PredictabilityOfAirfares} shows that airfares might be predictable to some extend, and do thus not follow these Brownian properties.

\citeA{maidanov2010monte} state in their article that the assumptions that the Black--Scholes model makes can never all be satisfied in a realistic setting. The authors argue that Monte~Carlo simulation can even evaluate European options more optimally in a realistic world than theoretical models. This is due to the fact that the Monte~Carlo method is extendible to any possible situation, and its only restrained by the computational capacity of the underlying hardware.

Valuation of American options proved to be much more difficult. \citeA[p.~408]{hull99} even stated that these types of options ``\textit{[\ldots] cannot easily handle situations where there are early exercise opportunities}''. This is due to the nature of the underlying decision model for executing the option. This path-dependency makes it very hard to evaluate the price of options, because holders can exercise their option an infinite number of times. \citeA{thompson1995valuation} states in his article that simulation approaches like Monte Carlo are best not employed when exercise decisions are present. However, many studies are available in which the researchers have developed reliable simulation models for this type of options (e.g., \citeA{garcia2003convergence}, \citeA{longstaff2001valuing}, \citeA{rogers2002monte}, \ldots).


\section{Predictability of airfares}
\label{sec:PredictabilityOfAirfares}
While there is much debate whether stocks follow a random distribution, ticket prices for flights might be predictable to a certain extend. Since 2000 the Trading Agent Competition (TAC) has submitted challenges to develop Artificial Intelligence-models allowing travel-related predictions \citeA{stone2005first}. While some of the models created in this competition seem to be fruitful (e.g., \citeA{ wellman2003price}), the simulation's circumstances are somewhat unrealistic. For example, the competition assumes an infinite capacity, and the demand distribution is stochastic with an upward trend.

The first to successfully apply an airfare prediction model on real data were \citeA{etzioni03}. During a 41~day period, the authors scraped pricing data on two~flights (i.e., LAX--BOS and SEA--IAD) every 3~hours. They then converted the acquired data into a time-series model and compared five different prediction methods on these data:
\begin{inparaenum}[\itshape (1)\upshape]
\item Time Series Forecasting;
\item Rule-based selection by hand;
\item Rule-based selection by an automated program;
\item Q-learning;
\item Hamlet.
\end{inparaenum}

The authors do not describe the \emph{Hamlet} approach in full, but a patent filed by \citeA{etzioni08} suggests that the model was build out of the best features of each of the four other approaches.

The models predicted whether future fares of a ticket were likely to rise or decline at a certain point in time. This prediction was then used to conclude if it was best to buy the ticket at instantly, or rather wait and hope for a better price. To compare the models with each other, \citeA{etzioni03} ran a simulation and analysed the saving generated by the method with the optimal savings for every flight. The authors showed that the best approach to this `buy-or-wait'--decision was the Hamlet model. The simulation stated that Hamlet could generate savings up to 60~percent of the optimal situation. The worst-performing method, the one based upon time-series forecasting, resulted in more than 80~percent extra costs compared to buying instantly.

\citeA{groves2013agent} later altered this model to predict the lowest price at a certain time in the future for \emph{any} flight and \emph{any} airline. They argue that this is more difficult due to the fact that aggregation of the data results in less variance.
The authors also compare their different approaches. They concluded that Partial Least Square regression (PLS) with automated feature selection is the most optimal savings-method resulting in a savings of 75~percent. \citeA{groves2013agent} based this automated feature selection upon a model build by the same authors \cite{groves2011regression}, which rationally and iteratively compared all different feature sets and computed the optimal solution.


